\documentclass[]{scrartcl}
\usepackage{Preamble}
\usepackage{amsmath}

\renewcommand{\exercise}{Exercise 5}
\renewcommand{\duedate}{2020-07-06, 18:00}

\begin{document}
    \section{Simulations}
    \subsection{4x4 Crossbar}
    \begin{verbatim}
        Bestimmen und begründen Sie wie die Werte zustande kommen
    \end{verbatim}

    \begin{equation}
        Avg = \sum_{n=0}^N \frac{\binom{N}{n}\left( N-n+1 \right)! \cdot C_n }{\sum_{k=0}^N \binom{N}{k}\left( N-k+1 \right)!}
    \end{equation}

    \begin{itemize}
        \item \verb|avg Input Queue Length|
            \begin{itemize}
                \item Generated Packets
                    \begin{itemize}
                        \item size: $ s = \SI{512}{b}$
                        \item send interval: $ t = \SI{512}{ns}$
                    \end{itemize}
                \item Connection to XBar
                    \begin{itemize}
                        \item data rate: $\SI{1}{Gbps}$
                    \end{itemize}
            \end{itemize}

        \begin{align}
            r_\text{gen} &= \frac{\SI{512}{b}}{\SI{512}{ns}} = \SI{1}{Gbps}\\
            C_n &= \SI{250}{Mbps}\cdot n\\
            r_\text{q fill,eff} &= \sum_{n=1}^4 \frac{\binom{4}{n}\left( 5-n \right)! \cdot C_n }{141}\\
                               &= \SI{347.5}{Mbps}\\
            \Rightarrow l_\text{q,avg} &= \frac{t_\text{sim}\cdot r_\text{q fill, eff}}{2\cdot s} = 3394
        \end{align}

        \item \verb|avg End-to-End Latency|
            \begin{itemize}
                \item (no delays inside buffers, because the generated data rate, even for all 4 apps, is lower than a single data rate channels maximum throughput)
                \item minimum
                    \begin{itemize}
                        \item App $\rightarrow$ C $\rightarrow$ Inport
                            $\rightarrow$ C $\rightarrow$ Outport $\rightarrow$ C $\rightarrow$ App
                        \item delay for packet: $t_{delay}=512ns$ (per \verb|DatarateChannel| C)
                            \begin{equation}
                                \Rightarrow t_{e2e,min} = \SI{1.536}{\mu s}
                            \end{equation}
                    \end{itemize}
                \item maximum

                    at end of simulation, inport buffer full, all in to one out

                    \begin{align}
                        l_{inport q, max} &= 7031\\
                        r_{dequeue, min} &= \SI{250}{Mbps}\\
                        t_\text{in queue, max} &= \frac{t_{q,inport}\cdot s}{r_{dequeue}} = \SI{3.599}{ms}\\
                    \end{align}

                \item on avg
                    \begin{align}
                        t_\text{e2e, avg} &= t_\text{in queue, max} / 2 = \SI{1.8}{ms}
                    \end{align}

            \end{itemize}
        \item \verb|avg Arbiter Request Queue Length|
            \begin{itemize}
                \item cases
                    \begin{align}
                        C_n &= n-1 \qquad 1\leq n\leq4=N
                    \end{align}
                \item on avg (analogously to $t_{e2e}$)
                    \begin{equation}
                        \Rightarrow l_{arbq,avg} = \sum_{n=1}^4 \frac{\binom{4}{n}\left( 5-n \right)! \cdot C_n }{141} = 0.39
                    \end{equation}
            \end{itemize}
        \item \verb|avg Arbiter Request Queue Time|
            \begin{itemize}
                \item cases
                    \begin{align}
                        C_n &= \SI{512}{ns} * n
                    \end{align}
                \item on avg
                    \begin{align}
                        \Rightarrow t_{arbq,avg} &= \sum_{n=1}^4 \frac{\binom{4}{n}\left( 5-n \right)! \cdot C_n }{141}\\
                                                 &= 
                    \end{align}
            \end{itemize}
        \item \verb|avg Output Buffer Queue Length|
            \begin{itemize}
                \item Generated Packets
                    \begin{itemize}
                        \item size: $ s = \SI{512}{b}$
                        \item send interval: $ t = \text{uniform}(\SI{1}{\mu s}, \SI{10}{\mu s}) = \SI{5.5}{\mu s}$
                    \end{itemize}
                \item Connection to and in XBar
                    \begin{itemize}
                        \item data rate: $\SI{1}{Gbps}$
                    \end{itemize}
            \end{itemize}
            \begin{align}
                r_\text{generated} =& \frac{\SI{512}{b}}{\SI{5.5}{\mu s}} = \SI{93}{Mbps}\\
                \Rightarrow l_\text{q,avg} = 0
            \end{align}
        \item \verb|avg Throughput|

            input and output buffers always empty
            \begin{align}
                r_{cross} &= 4*r_\text{gen}\\
                &= \SI{372}{Mbps}
            \end{align}
    \end{itemize}
    \subsection{Throughput vs. Ports}
    \begin{itemize}
        \item with ourr formula 
            {0.6, 0.44, 0.347518, 0.286498, 0.243324, 0.211277, 0.186604, 0.167047, 0.151176, 0.138045,

>    0.127005, 0.117594, 0.109478, 0.102407, 0.0961928, 0.0906881, 0.0857783, 0.0813722, 0.077396,

>    0.0737899, 0.0705046, 0.067499, 0.0647391, 0.0621958, 0.0598446, 0.0576646, 0.0556378, 0.0537485,

>    0.0519832, 0.0503302, 0.048779}
    \end{itemize}
    \subsection{Throughput vs. Injection Rate}
    \begin{itemize}
        \item not in saturation until delay $\leq \SI{832}{ns}$
        \item saturation pont $r_{sat} = \frac{\SI{512}{b}}{\SI{832}{ns}} = \SI{615}{Mbps}$
        \item The main reason will be the arbiter. It can only process packets at about \SI{600}{Mbps} on avg., therefore bottlenecking the rest of the system
    \end{itemize}
    \subsection{Throughput vs. Bandwidth}
    \begin{itemize}
        \item Network in aturation until Bandwith $> \SI{1600}{Mbps}$
        \item throughput at saturation point $r_{sat} = \SI{1}{Gbps}$
        \item If the Arbiter can only work at 62\% of the bandwidth, we reach this saturation point if 62\% of the bandwidth is \SI{1}{Gbps}, therefore the required bandwith is \SI{1613}{Gbps}
    \end{itemize}
\section{Optimizations}

\end{document}
